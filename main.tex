\documentclass[12pt,a4paper]{article}
\linespread{1.2}
\setlength{\parskip}{0.5em}
\setlength{\parindent}{1em}

\usepackage[french]{babel}
\usepackage{csquotes} % avant biblatex
\usepackage[backend=biber,style=authoryear,language=french,maxbibnames=10,natbib=true]{biblatex}
\addbibresource{references.bib}

\usepackage{graphicx, caption}
\usepackage{url}
\usepackage{amsmath, amssymb}
\usepackage{geometry}
\usepackage{setspace}
\usepackage{xcolor}
\usepackage[hidelinks,unicode]{hyperref}
\usepackage{enumitem, booktabs, multirow, lipsum}

\geometry{margin=2.5cm}
\setstretch{1.25}

\usepackage{fontspec}
\setmainfont{Arial}


% ====== Informations de la page de titre ======
\title{\textbf{Admixture et spéciation : perspectives génomiques sur le rôle du flux de gènes dans la diversification des espèces}}
\author{Florent LE QUELLEC}
\date{}

\begin{document}
\maketitle

\begin{abstract}
    % Résumé synthétique (10 lignes max) : présenter le thème, les enjeux, la problématique et la portée de la revue.

\end{abstract}

\tableofcontents
\newpage

% ==========================
\section{Introduction}
% Présenter le contexte général, la problématique et les enjeux.
% - Pourquoi la spéciation est un thème central de la biologie évolutive.
% - Pourquoi l’admixture (flux de gènes) complexifie la définition des espèces.
% - Objectif de la review : comprendre comment la génomique éclaire ces processus.

Depuis 1859, avec la publication de L’Origine des espèces, le monde de la biologie et de l’écologie a été profondément bouleversé par les travaux du naturaliste anglais Charles Darwin, qui proposa la théorie de l’évolution selon laquelle les espèces actuelles ont évolué au cours du temps à partir d’ancêtres communs \parencite{darwin1859origin}.
Un siècle plus tard, en 1953, la découverte de la structure de l’ADN par Rosalind Franklin, James Watson et Francis Crick a transformé notre compréhension du vivant, en reliant la génétique aux principes darwiniens \parencite{watson1953molecular}.
Cette révolution biologique a profondément modifié notre vision de la spéciation, en révélant que le flux génétique peut être à la fois un frein à la divergence des populations et un moteur de diversification, selon les contextes écologiques et évolutifs.

Il est important, avant d’aller plus loin, de comprendre cette notion de spéciation. Le terme apparaît pour la première fois en 1906 dans l’article Factors of Species-Formation rédigé par O. F. Cook, qui définit la spéciation comme le processus par lequel de nouvelles espèces se forment à partir d’ancêtres communs. Cook souligne que ce processus repose sur l’apparition de mécanismes en particulier environnementale conduisant à un isolement reproductif, autrement dit à l’interruption du flux de gènes entre populations \parencite{cook1906factors}.

Cependant, l’utilisation de la génomique montre que cette définition n’est pas tout à fait exacte. En effet, de nombreuses études mettent en évidence des phénomènes d’admixture entre espèces. L’admixture n’est rien d’autre qu’un flux de gènes, ce qui implique donc des reproductions entre des groupes censés être reproductivement isolés selon la définition de Cook.

Comprendre la façon dont le flux de gènes influence la spéciation constitue aujourd’hui un enjeu central en biologie évolutive, car il permet de mieux appréhender les mécanismes à l’origine de la diversification du vivant et de redéfinir les limites entre espèces.
Cette revue cherche ainsi à comprendre dans quelle mesure l’admixture, c’est-à-dire le flux de gènes entre lignées divergentes, constitue un frein ou un moteur de la spéciation.

Nous présenterons d’abord les concepts fondamentaux liés à la spéciation et à l’admixture, avant d’analyser le cas d’étude de l’article choisi portant sur les canidés nord-américains, puis nous élargirons la réflexion à d’autres systèmes biologiques et aux apports récents de la bioinformatique, en particulier de la génomique.
% ==========================
\section{Cadre théorique : spéciation et admixture}
% Présenter les concepts clés et la base théorique.
\subsection{Définitions et modèles classiques de la spéciation}
% Spéciation allopatrique, sympatrique, parapatrique, etc.
% Mécanismes d’isolement reproductif.
La spéciation est une notion clé en écologie évolutive : elle désigne le mécanisme par lequel se forment de nouvelles espèces. Introduit en 1906 par O. F. Cook, le terme a connu depuis d’importantes évolutions conceptuelles. Cook proposait une définition essentiellement cladistique, c’est-à-dire qu’il concevait la spéciation comme la division d’une lignée ancestrale en deux lignées distinctes, généralement sous l’effet de facteurs extérieurs tels que des changements environnementaux \parencite{cook1906factors}.

Environ trente ans plus tard, Theodosius Dobzhansky renouvelle profondément cette conception en réconciliant la théorie darwinienne de l’évolution et la génétique mendélienne.
Il montre que la spéciation n’est pas un simple incident de l’évolution, mais un processus évolutif à part entière, résultant de l’accumulation de différences génétiques conduisant à l’isolement reproductif entre populations.
Ainsi, la spéciation devient chez Dobzhansky une conséquence directe de l’évolution, pouvant être favorisée — mais non exclusivement causée — par des modifications de l’environnement \parencite{dobzhansky1951genetics}.

À la même période, Ernst Mayr propose une définition précise du terme espèce :

\begin{quote}
    \textit{“Species are groups of actually or potentially interbreeding natural populations, which are reproductively isolated from other such groups.”}
\end{quote}

Cette définition, encore largement utilisée aujourd’hui, a été complétée pour inclure la notion de descendance viable et féconde.
Mayr introduit également le concept de spéciation allopatrique, selon lequel la formation de nouvelles espèces résulte d’un isolement géographique durable \parencite{mayr1942systematics}.

Ainsi, une distinction s’impose entre les différents modèles de spéciation :

\begin{itemize}
    \item Spéciation allopatrique : elle repose sur un isolement géographique complet (séparation par une montagne, un rift, une mer, etc.), empêchant le flux de gènes entre les populations.
    \item Spéciation parapatrique : elle se produit entre des populations partiellement isolées, vivant dans des milieux aux conditions écologiques différentes. La sélection naturelle et la dérive génétique favorisent alors leur divergence malgré un flux génétique limité.
\end{itemize}

la notion de speciation se developpe encore, dans les années 60-80, les travaux sur la génomique avance, ce qui vient, modifier la vision des choses, on ne regarde non plus le phenotype, mais on y rajoute en plus le génotype dans la spéciation. Avec les travaux de J. Maynard Smith, un met en avant une nouvelle forme de spéciation:
\begin{itemize}
    \item La spéciation sympatrique : Sous des conditions spécifiques, la
          spéciation sympatrique est possible grâce à la sélection disruptive et à
          l'apparition d'isolements reproductifs dans un même espace géographique par
          polymorphisme génétique et comportemental.\parencite{maynardsmith1966sympatric}
\end{itemize}

Parallèlement, le développement de la phylogénétique moléculaire a profondément transformé l’étude de l’évolution.
L’analyse des séquences d’ADN permet désormais de reconstruire les relations évolutives entre espèces de façon quantitative. Les travaux pionniers de Joseph Felsenstein \parencite{felsenstein1981evolutionary} ont introduit des approches informatiques et statistiques fondées sur la vraisemblance, ouvrant la voie à une compréhension génomique de la divergence et de la spéciation.

Des travaux récents portant sur l’hybridation ont conduit à repenser la définition classique de la spéciation, en la rendant moins dépendante de la notion d’isolement reproductif strict. L’article de \textcite{schumer2018natural} illustre cette évolution en montrant la complexité du concept d’espèce et du processus de spéciation. Pour ces auteurs, la spéciation est un processus évolutif continu, au sein duquel des barrières reproductives partielles coexistent avec des échanges de gènes entre lignées.
L’isolement reproductif n’est donc pas absolu, et l’hybridation fait partie intégrante de l’histoire évolutive des espèces. Les auteurs concluent que l’hybridation ne contredit pas la spéciation ; au contraire, elle en révèle la nature dynamique et graduelle, dont les effets dépendent notamment du taux de recombinaison génétique au sein des génomes hybrides.

La synthèse de \textcite{Penalbaetal2024_HybridizationReview}, redéfinissent la spéciation comme un processus évolutif continu plutôt que comme un événement ponctuel. Les auteurs montrent que l’hybridation et l’admixture ne contredisent pas la spéciation, mais en constituent au contraire une composante dynamique. Le flux de gènes entre lignées divergentes peut tantôt ralentir la différenciation, tantôt favoriser l’émergence de nouvelles combinaisons adaptatives et même de nouvelles espèces. Ainsi, la spéciation est aujourd’hui perçue comme un continuum génomique, modulé par la sélection naturelle, la recombinaison et la structure du flux de gènes, où l’isolement reproductif est rarement absolu.

Ces avancées montrent que la spéciation n’est pas une rupture nette, mais un processus où la divergence évolutive peut coexister avec un certain flux de gènes.
Dans cette optique, l’admixture émerge comme un mécanisme central, capable à la fois de freiner la différenciation et de stimuler l’innovation génétique.
\subsection{L’admixture comme processus évolutif}
% Flux de gènes, hybridation, introgression adaptative.
% Effets possibles : homogénéisation vs innovation génétique.

Pour introduire la notion d’admixture, il est essentiel de comprendre celle de flux de gènes.
Le flux de gènes se définit comme un échange d’allèles entre deux populations, généralement au sein d’une même espèce.
Il contribue à maintenir la cohésion génétique entre populations en empêchant leur divergence.

L’admixture, quant à elle, désigne un flux de gènes entre des populations ou des espèces préalablement isolées.
Ce phénomène se produit souvent dans des \textit{zones hybrides}, c’est-à-dire des régions où deux espèces génétiquement proches entrent en contact et peuvent s’hybrider.  %rajouter bibliographie
Les hybrides issus de ces croisements sont le plus souvent stériles (incapables de transmettre leur génome) mais certains peuvent être viables et fertiles.
Dans ce cas, ils participent au transfert de matériel génétique entre espèces, un processus appelé introgression.
Lorsque les gènes introgressés confèrent un avantage sélectif, on parle alors d’introgression adaptative.

Un exemple bien documenté est celui de la souris domestique (\textit{Mus musculus domesticus}) et de la souris d’Afrique du Nord (\textit{Mus spretus}).
L’étude de \textcite{souris-orth-2002} met en évidence une introgression naturelle, bien que partielle, ayant permis le transfert d’un gène de résistance à certains pesticides vers des populations de souris domestiques.
Si un tel processus d’admixture persiste sur de longues périodes, il peut conduire à la formation d’une nouvelle espèce hybride, un phénomène appelé spéciation hybridique \parencite{vilaca-2023}.

Les flux génétiques peuvent ainsi avoir deux conséquences majeures :
d’une part, une homogénéisation des allèles entre populations, qui tend à réduire leur différenciation ;
et d’autre part, une innovation génétique, permettant l’émergence de nouvelles combinaisons adaptatives et l’accroissement de la diversité génétique au sein des espèces.

\subsection{Apports récents de la génomique}
% Comment les approches génomiques permettent d’estimer la divergence et le flux de gènes.
% Exemples de méthodes : D-statistics, MSMC2, ABC-RF, etc.

% ==========================
\section{Étude de cas : l’admixture chez les canidés nord-américains}

\subsection{Contexte et problématique}

La phylogénie des canidés nord-américains, notamment celle des loups, coyotes et hybrides présumés, constitue un cas emblématique des difficultés à délimiter les espèces lorsque le flux de gènes persiste après la divergence.
Depuis plusieurs décennies, la validité du statut spécifique du \textit{loup de l’Est} (\textit{Canis lycaon}) fait l’objet d’un débat intense : s’agit-il d’une espèce distincte, d’une sous-espèce du loup gris (\textit{C. lupus}) ou d’un hybride stable entre le loup gris et le coyote (\textit{C. latrans}) ?
Cette question n’est pas qu’académique : elle conditionne les politiques de conservation, notamment au Canada et aux États-Unis, où le statut juridique des populations dépend de leur reconnaissance taxonomique.

\subsection{Données et approches utilisées \parencite{vilaca-2023}}

Pour tester les hypothèses concurrentes sur l’origine du loup de l’Est, \textcite{vilaca-2023} ont analysé des génomes entiers issus de plusieurs populations de loups, coyotes et hybrides.
L’étude s’appuie sur des approches de coalescence et de simulation génomique permettant d’estimer à la fois les temps de divergence et les proportions d’admixture entre lignées.
Les auteurs ont notamment mobilisé :
\begin{itemize}
    \item des tests d’introgression fondés sur les \textit{D-statistics} (tests ABBA-BABA) pour quantifier les flux de gènes récents et anciens ;
    \item la méthode MSMC2 (\textit{Multiple Sequentially Markovian Coalescent}) pour reconstruire l’histoire démographique des lignées ;
    \item et des approches de modélisation bayésienne (ABC-RF) pour comparer différents scénarios de divergence avec ou sans flux de gènes.
\end{itemize}

\subsection{Résultats principaux}

Les analyses confirment que les trois lignées principales (\textit{C. lupus}, \textit{C. lycaon} et \textit{C. latrans}) ont divergé il y a environ 60 à 70 000 ans, mais que des événements d’admixture multiples ont eu lieu depuis cette séparation.
En particulier, le loup de l’Est apparaît comme une lignée distincte, ayant connu des échanges génétiques anciens avec le coyote.
Les tests de modèles évolutifs montrent qu’un scénario à trois espèces distinctes, mais connectées par des flux de gènes post-divergence, explique mieux les données qu’un modèle à deux espèces avec hybridation récente.
Ces résultats indiquent une mosaïque génomique complexe, où certaines régions du génome reflètent la divergence ancienne, tandis que d’autres conservent des signatures d’introgression récente.

\subsection{Portée des résultats}

Cette étude remet en cause la vision dichotomique selon laquelle hybridation et spéciation seraient incompatibles.
Elle illustre comment l’admixture peut contribuer à la diversification, en permettant l’introduction de variations adaptatives sans effacer les barrières d’isolement reproductif.
Sur le plan méthodologique, elle démontre la puissance des approches génomiques intégrées (coalescence, tests d’introgression et inférences bayésiennes) pour démêler les histoires évolutives réticulées.
Enfin, ce cas met en lumière les enjeux pratiques : dans les canidés, la distinction entre hybrides récents et lignées stabilisées a des conséquences directes sur la gestion de la biodiversité et la protection légale des populations.



% ==========================
\section{Comparaisons avec d’autres systèmes biologiques}
% Montrer que ces processus sont généraux.
\subsection{Admixture chez les humains et Néandertaliens}
\subsection{Hybridation adaptative chez les oiseaux et poissons}
\subsection{Flux de gènes et diversification chez les plantes}
% Chaque sous-section illustre un cas d’admixture et ses conséquences évolutives.

% ==========================
\section{Apports et limites des approches génomiques}
% Évaluer les méthodes, leurs forces et leurs biais.
\subsection{Forces : puissance de résolution, datation fine, reconstruction historique}
\subsection{Limites : couverture, échantillonnage, interprétation des signaux}
\subsection{Pistes futures : ADN ancien, intégration éco-évo, modélisation avancée}

% ==========================
\section{Synthèse et perspectives}
% Discussion générale : ce qu’on retient, les grandes tendances, les questions ouvertes.
L’ensemble des travaux étudiés souligne que l’admixture n’est pas une exception mais un processus omniprésent.
Elle peut retarder ou accélérer la spéciation, selon le contexte écologique et génétique.

% ==========================
\printbibliography

\section*{Licence / License}
\begin{center}
    \includegraphics[width=2.5cm]{figures/cc-by-nc.png}\\[0.5em]
    \textbf{Ce document est distribué sous licence\\
        Creative Commons Attribution–NonCommercial 4.0 International (CC BY–NC 4.0).}\\[0.5em]
    Vous êtes libre de le partager et de l’adapter à condition d’en attribuer la paternité\\
    et de ne pas en faire un usage commercial.\\[0.3em]
    \url{https://creativecommons.org/licenses/by-nc/4.0/}\\[1em]
    \textbf{This document is distributed under the\\
        Creative Commons Attribution–NonCommercial 4.0 International License (CC BY–NC 4.0).}\\[0.5em]
    You are free to share and adapt the material, provided that appropriate credit is given\\
    and it is not used for commercial purposes.\\[0.3em]
    \url{https://creativecommons.org/licenses/by-nc/4.0/}
\end{center}


\end{document}
