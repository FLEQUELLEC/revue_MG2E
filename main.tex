\documentclass[12pt,a4paper]{article}

% ============================
% ENCODAGE & LANGUE
% ============================
\usepackage{fontspec}             % gestion moderne des polices (XeLaTeX obligatoire)
\usepackage[french]{babel}
\babelfont{rm}{TeX Gyre Heros}    % équivalent libre d’Arial (avec petites capitales)
\babelfont{sf}{TeX Gyre Heros}
\babelfont{tt}{TeX Gyre Cursor}

% ============================
% MISE EN PAGE
% ============================
\usepackage[a4paper,margin=2.5cm]{geometry}
\usepackage{setspace}
\onehalfspacing                   % interligne 1,5
\setlength{\parskip}{0.5em}
\setlength{\parindent}{1em}
\usepackage{booktabs, multirow}
\usepackage{enumitem}
\setlist{noitemsep}

% ============================
% MATH, COULEURS, SYMBOLES
% ============================
\usepackage{amsmath, amssymb}
\usepackage{xcolor}

% ============================
% IMAGES & TABLEAUX
% ============================
\usepackage{graphicx}
\usepackage{caption}
\captionsetup{labelfont=bf}

% ============================
% LIENS HYPERTEXTES
% ============================
\usepackage[hidelinks,unicode]{hyperref}
\hypersetup{
    colorlinks=true,
    linkcolor=black,
    urlcolor=blue,
    citecolor=teal,
    pdfauthor={Florent Le Quellec},
    pdftitle={Revue MG2E – Admixture & Spéciation},
    pdfsubject={Master BI/BEE – MG2E},
    pdfkeywords={spéciation, admixture, hybridation, génomique, bioinformatique}
}

% ============================
% CITATIONS & BIBLIOGRAPHIE
% ============================
\usepackage{csquotes} % doit être avant biblatex
\usepackage[
    backend=biber,
    style=authoryear,
    citestyle=authoryear-comp,
    sorting=nyt,
    maxcitenames=2,
    maxbibnames=99,
    giveninits=true,
    uniquename=false,
    url=false,
    doi=true,
    eprint=false,
    language=french
]{biblatex}

\DefineBibliographyStrings{french}{
    andothers = {et al.}
}

\addbibresource{references.bib}


% ============================
% DIVERS / AJUSTEMENTS
% ============================
\frenchspacing   % espace uniforme après les points
\sloppy          % évite les dépassements de marges
\setstretch{1.25}

% ====== Informations de la page de titre ======
% ==========================
% PAGE DE TITRE PERSONNALISÉE
% ==========================
\begin{document}

\begin{titlepage}
    \begin{center}
        % ---- Ligne supérieure avec logos ----
        \begin{minipage}{0.45\textwidth}
            \begin{flushleft}
                \includegraphics[width=4cm]{figures/logo_UT.png}
            \end{flushleft}
        \end{minipage}
        \hfill
        \begin{minipage}{0.45\textwidth}
            \begin{flushright}
                \includegraphics[width=6cm]{figures/logobgesansfond.png}
            \end{flushright}
        \end{minipage}

        \vspace{2.5cm}

        % ---- Titre principal ----
        {\Large \textbf{Université Toulouse III – Paul Sabatier}}\\[0.5em]
        {\large Master BI/BEE – Mention Bioinformatique et Génomique Environnementale}\\[2cm]

        {\Huge \textbf{Admixture et spéciation :}}\\[0.3em]
        {\LARGE \textbf{Perspectives génomiques sur le rôle du flux de gènes dans la diversification des espèces}}\\[2cm]

        % ---- Auteur ----
        {\Large Florent \textsc{LE QUELLEC}}\\[1.5cm]

        % ---- Encadrants ou UE ----
        {\large Revue scientifique dans le cadre de l’UE \textit{Communication scientifique – MG2E}}\\[0.5em]
        {\large Année universitaire 2025–2026}\\[2cm]

        % ---- Licence CC ----
        \includegraphics[width=2cm]{figures/cc-by-nc.png}\\[0.5em]
        {\small Licence \textbf{Creative Commons BY–NC 4.0}}\\
        \url{https://creativecommons.org/licenses/by-nc/4.0/}

        \vfill
    \end{center}
\end{titlepage}


\begin{abstract}
    % Résumé synthétique (10 lignes max) : présenter le thème, les enjeux, la problématique et la portée de la revue.
    La spéciation, processus clé de la diversification du vivant, repose sur l’établissement progressif de barrières reproductives entre populations. Cependant, les avancées récentes en génomique ont révélé que le flux de gènes (admixture) persiste souvent bien au-delà de la divergence initiale, rendant floue la frontière entre espèces. Cette revue examine comment l’admixture peut agir à la fois comme frein et moteur de la spéciation, selon les contextes écologiques et évolutifs. Après un rappel des modèles classiques et de leurs révisions contemporaines, nous présentons les apports méthodologiques des approches génomiques et bio-informatiques, qui permettent désormais de quantifier le flux de gènes avec une précision sans précédent (tests d’introgression, coalescence, inférences bayésiennes). L’étude de cas sur les canidés nord-américains illustre la complexité des histoires évolutives réticulées, où hybridation et divergence coexistent. En comparaison, les exemples humains et végétaux soulignent le caractère généralisé de ces processus. Enfin, les limites des approches actuelles (couverture inégale, biais d’échantillonnage, interprétation des signaux) appellent à une intégration accrue des données écologiques et des modèles évolutifs. L’admixture apparaît ainsi non comme une exception, mais comme une composante essentielle de la dynamique de spéciation.

    \vspace{1em}
    \noindent\textbf{Mots-clés :} spéciation, admixture, hybridation, flux de gènes, génomique, bioinformatique, introgression, continuum de spéciation
\end{abstract}

\tableofcontents
\newpage

% ==========================
\section{Introduction}
% Présenter le contexte général, la problématique et les enjeux.
% - Pourquoi la spéciation est un thème central de la biologie évolutive.
% - Pourquoi l’admixture (flux de gènes) complexifie la définition des espèces.
% - Objectif de la review : comprendre comment la génomique éclaire ces processus.

Depuis 1859, avec la publication de L’Origine des espèces, le monde de la biologie et de l’écologie a été profondément bouleversé par les travaux du naturaliste anglais Charles Darwin, qui proposa la théorie de l’évolution selon laquelle les espèces actuelles ont évolué au cours du temps à partir d’ancêtres communs \parencite{darwin1859origin}.
Un siècle plus tard, en 1953, la découverte de la structure de l’ADN par Rosalind Franklin, James Watson et Francis Crick a transformé notre compréhension du vivant, en reliant la génétique aux principes darwiniens \parencite{watson1953molecular}.
Cette révolution biologique a profondément modifié notre vision de la spéciation, en révélant que le flux génétique peut être à la fois un frein à la divergence des populations et un moteur de diversification, selon les contextes écologiques et évolutifs.

Il est important, avant d’aller plus loin, de comprendre cette notion de spéciation. Le terme apparaît pour la première fois en 1906 dans l’article Factors of Species-Formation rédigé par O. F. Cook, qui définit la spéciation comme le processus par lequel de nouvelles espèces se forment à partir d’ancêtres communs. Cook souligne que ce processus repose sur l’apparition de mécanismes en particulier environnementale conduisant à un isolement reproductif, autrement dit à l’interruption du flux de gènes entre populations \parencite{cook1906factors}.

Cependant, l’utilisation de la génomique remet en cause la définition proposé par Cook. En effet, de nombreuses études mettent en évidence des phénomènes d’admixture entre espèces. L’admixture n’est rien d’autre qu’un flux de gènes, ce qui implique donc des reproductions entre des groupes censés être reproductivement isolés selon la définition de Cook.

Comprendre la façon dont le flux de gènes influence la spéciation constitue aujourd’hui un enjeu central en biologie évolutive, car il permet de mieux appréhender les mécanismes à l’origine de la diversification du vivant et de redéfinir les limites entre espèces.
Cette revue cherche ainsi à comprendre dans quelle mesure l’admixture, c’est-à-dire le flux de gènes entre lignées divergentes, constitue un frein ou un moteur de la spéciation.

Nous présenterons d’abord les concepts fondamentaux liés à la spéciation et à l’admixture, avant d’analyser le cas d’étude de l’article choisi portant sur les canidés nord-américains, puis nous élargirons la réflexion à d’autres systèmes biologiques et aux apports récents de la bioinformatique, en particulier de la génomique.
% ==========================
\section{Cadre théorique : spéciation et admixture}
% Présenter les concepts clés et la base théorique.
\subsection{Définitions et modèles classiques de la spéciation}
% Spéciation allopatrique, sympatrique, parapatrique, etc.
% Mécanismes d’isolement reproductif.
La spéciation est une notion clé en écologie évolutive : elle désigne le mécanisme par lequel se forment de nouvelles espèces. Introduit en 1906 par O. F. Cook, le terme a connu depuis d’importantes évolutions conceptuelles. Cook proposait une définition essentiellement cladistique, c’est-à-dire qu’il concevait la spéciation comme la division d’une lignée ancestrale en deux lignées distinctes, généralement sous l’effet de facteurs extérieurs tels que des changements environnementaux \parencite{cook1906factors}.

Environ trente ans plus tard, Theodosius Dobzhansky renouvelle profondément cette conception en réconciliant la théorie darwinienne de l’évolution et la génétique mendélienne.
Il montre que la spéciation n’est pas un simple incident de l’évolution, mais un processus évolutif à part entière, résultant de l’accumulation de différences génétiques conduisant à l’isolement reproductif entre populations.
Ainsi, la spéciation devient chez Dobzhansky une conséquence directe de l’évolution, pouvant être favorisée (mais non exclusivement causée) par des modifications de l’environnement \parencite{dobzhansky1951genetics}.

À la même période, Ernst Mayr propose une définition précise du terme espèce :

\begin{quote}
    \textit{“Species are groups of actually or potentially interbreeding natural populations, which are reproductively isolated from other such groups.”}
\end{quote}

Cette définition, encore largement utilisée aujourd’hui, a été complétée pour inclure la notion de descendance viable et féconde.
Mayr introduit également le concept de spéciation allopatrique, selon lequel la formation de nouvelles espèces résulte d’un isolement géographique durable \parencite{mayr1942systematics}.

Ainsi, une distinction s’impose entre les différents modèles de spéciation :

\begin{itemize}
    \item Spéciation allopatrique : elle repose sur un isolement géographique complet (séparation par une montagne, un rift, une mer, etc.), empêchant le flux de gènes entre les populations.
    \item Spéciation parapatrique : elle se produit entre des populations partiellement isolées, vivant dans des milieux aux conditions écologiques différentes. La sélection naturelle et la dérive génétique favorisent, la divergence malgré un flux génétique limité.
\end{itemize}

La notion de spéciation continue d’évoluer. Entre les années 1960 et 1980, les progrès de la génétique et de la génomique ont profondément modifié la compréhension du processus : l’analyse ne se limite plus au phénotype, mais intègre désormais le génotype. Les travaux de J. Maynard Smith ont notamment mis en avant une nouvelle forme de spéciation :
\begin{itemize}
    \item La spéciation sympatrique : sous certaines conditions, la spéciation sympatrique peut survenir grâce à la sélection disruptive et à l’apparition d’isolements reproductifs au sein d’une même population, dans un espace géographique partagé, par l’intermédiaire de polymorphismes génétiques et comportementaux \parencite{maynardsmith1966sympatric}.
\end{itemize}


Parallèlement, le développement de la phylogénétique moléculaire a profondément transformé l’étude de l’évolution.
L’analyse des séquences d’ADN permet désormais de reconstruire les relations évolutives entre espèces de façon quantitative. Les travaux pionniers de Joseph Felsenstein \parencite{felsenstein1981evolutionary} ont introduit des approches informatiques et statistiques fondées sur la vraisemblance, ouvrant la voie à une compréhension génomique de la divergence et de la spéciation.

Des travaux récents portant sur l’hybridation ont conduit à repenser la définition classique de la spéciation, en la rendant moins dépendante de la notion d’isolement reproductif strict. L’article de \textcite{schumer2018natural} illustre cette évolution en montrant la complexité du concept d’espèce et du processus de spéciation. Pour ces auteurs, la spéciation est un processus évolutif continu, au sein duquel des barrières reproductives partielles coexistent avec des échanges de gènes entre lignées.
L’isolement reproductif n’est donc pas absolu, et l’hybridation fait partie intégrante de l’histoire évolutive des espèces. Les auteurs concluent que l’hybridation ne contredit pas la spéciation ; au contraire, elle en révèle la nature dynamique et graduelle, dont les effets dépendent notamment du taux de recombinaison génétique au sein des génomes hybrides.

La conception même de la spéciation est en train de changer. Selon une synthèse récente, il ne s'agirait plus d'un événement ponctuel, mais d'un processus évolutif continu \parencite{Penalbaetal2024_HybridizationReview}. Dans ce nouveau cadre, l'hybridation et l'admixture ne sont plus de simples anomalies ; elles deviennent une composante dynamique de la formation des espèces. Loin d'être une force uniquement stabilisatrice, le flux de gènes entre lignées divergentes se révèle être un acteur ambivalent : il peut tantôt ralentir la différenciation, tantôt favoriser l'émergence de nouvelles combinaisons adaptatives. L'admixture émerge ainsi comme un mécanisme central, capable à la fois de freiner la différenciation et de stimuler l'innovation génétique.
La spéciation n'est donc plus une rupture nette, mais un continuum où la divergence évolutive peut, de manière contre-intuitive, coexister avec un flux de gènes persistant.

\subsection{L’admixture comme processus évolutif}
% Flux de gènes, hybridation, introgression adaptative.
% Effets possibles : homogénéisation vs innovation génétique.

Pour introduire la notion d’admixture, il est essentiel de comprendre celle de flux de gènes.
Le flux de gènes se définit comme un échange d’allèles entre deux populations, généralement au sein d’une même espèce.
Il contribue à maintenir la cohésion génétique entre populations en empêchant leur divergence.

L’admixture, quant à elle, désigne un flux de gènes entre des populations ou des espèces préalablement isolées.
Ce phénomène se produit souvent dans des \textit{zones hybrides}, c’est-à-dire des régions où deux espèces génétiquement proches entrent en contact et peuvent s’hybrider.  %rajouter bibliographie
Les hybrides issus de ces croisements sont le plus souvent stériles (incapables de transmettre leur génome) mais certains peuvent être viables et fertiles.
Dans ce cas, ils participent au transfert de matériel génétique entre espèces, un processus appelé introgression.
Lorsque les gènes introgressés confèrent un avantage sélectif, on parle alors d’introgression adaptative.

Un exemple bien documenté est celui de la souris domestique (\textit{Mus musculus domesticus}) et de la souris d’Afrique du Nord (\textit{Mus spretus}).
L’étude de \textcite{souris-orth-2002} met en évidence une introgression naturelle, bien que partielle, ayant permis le transfert d’un gène de résistance à certains pesticides vers des populations de souris domestiques.
Si un tel processus d’admixture persiste sur de longues périodes, il peut conduire à la formation d’une nouvelle espèce hybride, un phénomène appelé spéciation hybridique \parencite{vilaca-2023}.

Les flux génétiques peuvent ainsi avoir deux conséquences majeures.
D’une part, une homogénéisation des allèles entre populations, qui tend à réduire leur différenciation et constitue donc un frein à la spéciation.
Par exemple, chez certaines espèces migratrices qui se reproduisent dans des zones communes, le brassage génétique important maintient une forte connectivité entre populations et empêche leur divergence.
D’autre part, les flux génétiques peuvent être sources d’innovation génétique, en favorisant l’émergence de nouvelles combinaisons adaptatives et l’accroissement de la diversité génétique au sein des espèces.
Cet effet, inverse de l’homogénéisation, agit alors comme un moteur de spéciation : les introgressions génétiques peuvent différencier suffisamment le génome des hybrides de celui des espèces parentales pour conduire à la formation de nouvelles lignées.

\subsection{Apports récents de la génomique et de la bio-informatique}
% Comment les approches génomiques permettent d’estimer la divergence et le flux de gènes.
% Exemples de méthodes : D-statistics, MSMC2, ABC-RF, etc.

Les progrès de la génomique et de la bio-informatique ont profondément renouvelé notre compréhension des processus de spéciation et d’admixture.
Les approches modernes ne se limitent plus à la comparaison de quelques marqueurs moléculaires, mais exploitent désormais des génomes complets pour estimer à la fois la divergence évolutive et le flux de gènes entre populations.
Ces analyses permettent d’identifier les régions du génome impliquées dans l’isolement reproductif, de détecter les signatures d’introgression et de quantifier l’intensité du flux génétique au cours du temps \parencite{seehausen2014genomics}.

Parmi les outils les plus couramment employés pour détecter un mélange génétique entre lignées, on trouve la statistique D, aussi connue sous le nom de test ABBA–BABA. Initialement mise au point pour identifier les traces d’introgression entre humains modernes et Néandertaliens, cette méthode compare les fréquences d’allèles partagés entre quatre groupes (ou taxons).
Lorsqu’elle révèle un déséquilibre significatif dans la répartition de ces allèles, cela suggère que le schéma de divergence n’a pas suivi une bifurcation évolutive strictement indépendante, autrement dit, qu’il y a eu échange de gènes entre lignées \parencite{green2010neandertal, durand2011testing}.
Des développements récents, comme les tests f4-ratio ou f-branch, permettent même d’estimer la proportion d’ADN héritée d’une autre espèce, offrant ainsi une vision quantitative du degré d’introgression.

Au-delà de ces approches ponctuelles, les chercheurs s’appuient aujourd’hui sur des modèles de coalescence pour reconstruire la chronologie du flux de gènes entre populations.
Par exemple, la méthode MSMC2 (Multiple Sequentially Markovian Coalescent) permet de retracer, à partir de génomes complets, les variations du taux de coalescence dans le temps.
Elle offre une estimation fine du moment où deux lignées ont cessé d’échanger des gènes, et a permis de révéler des épisodes anciens d’admixture, aussi bien chez les humains que chez d’autres vertébrés \parencite{schiffels2014msmc}.

En parallèle, les approches bayésiennes et de simulation se sont imposées comme des outils clés pour tester différents scénarios de spéciation.
La méthode dite Approximate Bayesian Computation (ABC) simule des jeux de données sous diverses hypothèses d’isolement ou de flux génétique, puis compare les résultats aux données observées pour identifier le scénario le plus plausible.
L’intégration de l’apprentissage automatique, notamment sous la forme de forêts aléatoires dans le cadre de l’ABC-Random Forest (ABC-RF), a considérablement renforcé la robustesse et la précision de ces analyses \parencite{pudlo2016abc, raynal2019abc}.

Enfin, l’association de ces outils avec des approches de cartographie génomique et de détection de sélection naturelle (telles que les statistiques $F_{ST}$, $d_{XY}$ ou les analyses de déséquilibre de liaison) offre une lecture plus fine des processus évolutifs.
Les génomes apparaissent alors comme des mosaïques, où certaines régions sont fortement différenciées — souvent en lien avec des barrières reproductives, tandis que d’autres restent perméables au flux de gènes.
Cette vision, qualifiée de « continuum de spéciation », met en lumière la nature dynamique du processus évolutif, où divergence, hybridation et sélection naturelle interagissent de façon continue \parencite{roux2016continuum, kulmuni2020towards}.

En somme, la génomique moderne et la bio-informatique ont profondément transformé notre compréhension de la spéciation.
Elles permettent non seulement de mesurer et visualiser le flux de gènes, mais aussi de relier la structure du génome aux forces évolutives qui façonnent la diversification du vivant.


% ==========================

\section{Étude de cas : l’admixture chez les canidés nord-américains}
\subsection{Contexte et problématique}

La phylogénie des canidés nord-américains, notamment celle des loups, coyotes et hybrides présumés, constitue un cas important des difficultés à délimiter les espèces lorsque le flux de gènes persiste après la divergence.
Depuis plusieurs décennies, la validité du statut spécifique du loup de l’Est (\textit{Canis lycaon}) fait l’objet d’un débat intense : s’agit-il d’une espèce distincte, d’une sous-espèce du loup gris (\textit{C. lupus}) ou d’un hybride stable entre le loup gris et le coyote (\textit{C. latrans}) ?
Cette question n’est pas qu’académique : elle influence les politiques de conservation, notamment au Canada et aux États-Unis, où le statut juridique des populations dépend de leur reconnaissance taxonomique.

\subsection{Données et approches utilisées \parencite{vilaca-2023}}

Pour tester les hypothèses sur l’origine du loup de l’Est, \textcite{vilaca-2023} ont analysé des génomes issus de plusieurs populations de loups, coyotes et hybrides.
L’étude s’appuie sur des approches de coalescence et de simulation génomique permettant d’estimer à la fois les temps de divergence et les proportions d’admixture entre lignées.
Les auteurs ont notamment mobilisé :
\begin{itemize}
    \item des tests d’introgression fondés sur les \textit{D-statistics} (tests ABBA-BABA) pour quantifier les flux de gènes récents et anciens ;
    \item la méthode MSMC2 (\textit{Multiple Sequentially Markovian Coalescent}) pour reconstruire l’histoire démographique des lignées ;
    \item et des approches de modélisation bayésienne (ABC-RF) pour comparer différents scénarios de divergence avec ou sans flux de gènes.
\end{itemize}

\subsection{Résultats principaux}

Les analyses confirment que les trois lignées principales (\textit{C. lupus}, \textit{C. lycaon} et \textit{C. latrans}) ont divergé il y a environ 60 à 70 000 ans, mais que des événements d’admixture multiples ont eu lieu depuis cette séparation.
En particulier, le loup de l’Est apparaît comme une lignée distincte, ayant connu des échanges génétiques anciens avec le coyote.
Les tests de modèles évolutifs montrent qu’un scénario à trois espèces distinctes, mais connectées par des flux de gènes post-divergence, explique mieux les données qu’un modèle à deux espèces avec hybridation récente.
Ces résultats indiquent une mosaïque génomique complexe, où certaines régions du génome reflètent la divergence ancienne, tandis que d’autres conservent des signatures d’introgression récente.

\subsection{Portée des résultats}

Cette étude remet en cause la double vision selon laquelle hybridation et spéciation seraient incompatibles.
Elle illustre comment l’admixture peut contribuer à la diversification, en permettant l’introduction de variations adaptatives sans effacer les barrières d’isolement reproductif.
Sur le plan méthodologique, elle démontre la puissance des approches génomiques intégrées (coalescence, tests d’introgression et inférences bayésiennes) pour démêler les histoires évolutives réticulées.
Enfin, ce cas met en lumière les enjeux pratiques : dans les canidés, la distinction entre hybrides récents et lignées stabilisées a des conséquences directes sur la gestion de la biodiversité et la protection légale des populations.

\section{Comparaisons avec d’autres systèmes biologiques}
% Montrer que ces processus sont généraux.
\subsection{Admixture chez les humains et Néandertaliens}
Le genre \textit{Homo} n'échappe pas aux processus d’admixture.
Des études génomiques majeures ont révélé que les humains modernes (\textit{Homo sapiens}) ont hérité d’environ 1 à 2\,\% de leur ADN des Néandertaliens (\textit{Homo neanderthalensis}), une espèce humaine éteinte ayant coexisté avec eux en Eurasie jusqu’à il y a environ 40\,000 ans \parencite{green2010neandertal}.

Ces travaux, fondés sur le séquençage complet du génome néandertalien, ont mis en évidence un flux de gènes ancien entre lignées humaines autrefois considérées comme strictement séparées. \parencite{prufer2014complete}

L’admixture a eu lieu après la sortie d’Afrique des populations de \textit{H. sapiens}, probablement au Proche-Orient ou en Europe, et a contribué à l’introduction de variants génétiques impliqués dans l’immunité, la kératinisation ou encore l’adaptation aux climats froids.

Ce cas emblématique illustre que l’isolement reproductif entre espèces proches peut être perméable, même chez les hominidés, et que l’admixture peut façonner l’évolution adaptative plutôt que la contrecarrer.
Il constitue également un jalon historique dans l’étude de la spéciation, puisque c’est grâce à ces analyses que des méthodes comme la \textit{D-statistic} (test ABBA–BABA) ont été développées pour détecter l’introgression génétique entre lignées divergentes \parencite{durand2011testing}.

\subsection{Flux de gènes et diversification chez les plantes}
Les plantes offrent un autre exemple interesant de l’impact de l’admixture sur la spéciation. c'est l'admixture qui a permis cette diversification de plante agricole. Par exemple, le maïs (\textit{Zea mays}) est issu d’un processus complexe d’hybridation entre différentes espèces sauvages du genre \textit{Zea}, notamment le téosinte (\textit{Zea mays ssp. parviglumis}).
Ou encore le blé (\textit{Triticum aestivum}) qui est un hybride allopolyploïde résultant de croisements entre plusieurs espèces de graminées, notamment \textit{Triticum urartu}, \textit{Aegilops speltoides} et \textit{Aegilops tauschii} \parencite{feldman2001origin, feldman2007domestication}.

Ces hybridations ont permis l’introduction de gènes conférant des traits agronomiques avantageux, tels que la résistance aux maladies, l’adaptation aux conditions climatiques variées et l’amélioration des rendements.
L’admixture a ainsi joué un rôle crucial dans la domestication et l’amélioration des plantes cultivées, en favorisant la diversité génétique nécessaire à l’adaptation aux environnements agricoles
\section{Apports et limites des approches génomiques}
% Évaluer les méthodes, leurs forces et leurs biais.
\subsection{Forces : puissance de résolution, datation fine, reconstruction historique}

Il est indéniable que les approches génomiques ont révolutionné l’étude de la spéciation et de l’admixture. En permettant d'acceder à la totalité des génomes, offrant une vision puissante de la diversité génétique, du flux de genes et des mécanismes adaptatifs sous-jacents.

Contrairement aux marqueurs moléculaires traditionnels, qui ne capturent qu'une fraction limitée de la variation génétique, les génomes entiers permettent d'identifier des signatures fines d'introgression et de divergence. la bioinformatique a permit de cartographier ces génomes, donc de discriminer a cette échelle, révélant des zones de forte différenciation (barrières reproductives) et des régions perméables au flux de gènes, donc des régions associées a des genes sous selection. \parencite{seehausen2014genomics,roux2016continuum}

les méthodes modernes, telles que les D-statistics, MSMC2 et ABC-RF, offrent une capacité sans précédent à dater les événements d’admixture et à reconstruire l’histoire démographique des populations. Elles permettent de quantifier précisément les proportions d’introgression, de modéliser des scénarios complexes de spéciation avec flux de gènes, et d’estimer les temps de divergence avec une résolution temporelle fine \parencite{schiffels2014msmc,pudlo2016abc}.
Les methodes bayésiennes et de simulation permettent également de tester rigoureusement des sénario évolutifs, en intégrant l’incertitude et la variabilité des données génomiques, permttant d'estimer la vraisemblance des modèles proposés. \parencite{pudlo2016abc,raynal2019abc}

Combinées, elles offrent une puissance d'analyse pour relier la structure du génome, l’histoire démographique et les forces évolutives responsables de la diversification du vivant.

Cependant, cette section examine les principaux apports, les limites méthodologiques et les perspectives futures de ces approches.*


\subsection{Limites : couverture, échantillonnage, interprétation des signaux}

Malgré leurs avancées, les approches génomiques présentent plusieurs limites méthodologiques importantes. Tout d’abord, la qualité et la couverture des données génomiques influencent fortement la fiabilité des inférences. Des séquences incomplètes ou de faible profondeur peuvent introduire des biais dans l’estimation des fréquences alléliques, affectant ainsi les tests d’introgression et les modèles démographiques \parencite{han2014characterizing}. De plus l'échantillonnage des population est limité, pouvant conduire una sous estimations de la diversité de nos résulats, en particulier des especes éloignées des especes modeles.

la complexité des histoires évolutives, souvent marquées par des événements multiples d’admixture, de sélection et de dérive génétique, complique l’interprétation des signaux génomiques. Par exemple, des signatures d’introgression peuvent être confondues avec des effets de convergence adaptative ou de variation démographique, rendant difficile la distinction entre flux de gènes et autres processus évolutifs \parencite{joly2009statistical}.

\subsection{Pistes futures : ADN ancien, intégration éco-évo, modélisation avancée}
Les perspectives qui visent à dépasser les limites actuelles en intégrant de nouvelles sources de données et des cadres analytiques plus réalistes constituent une voie prometteuse.
L’étude de l’ADN ancien permet la possibilité d’observer directement les flux de gènes au cours du temps, en retraçant les événements d’admixture passés plutôt que de les inférer uniquement à partir des génomes contemporains \parencite{green2010neandertal, prufer2014complete}.
Ces données temporelles, combinées aux modèles de coalescence, permettent d’estimer plus finement la durée et la direction des flux génétiques.

Parallèlement, l’intégration de la dimension écologique et des modèles éco-évolutifs (qui relient environnement, sélection et démographie) constitue une voie prometteuse.
Ces approches visent à comprendre comment les gradients écologiques, les pressions de sélection locales et la connectivité des habitats façonnent la mosaïque génomique de la spéciation.
Enfin, les développements récents en modélisation bayésienne hiérarchique et en apprentissage profond (deep learning évolutif) laissent entrevoir une automatisation partielle de l’inférence démographique, tout en intégrant la complexité des paysages génétiques et environnementaux.

En somme, les approches génomiques ont profondément transformé notre compréhension de la spéciation et de l’admixture.
Elles offrent une vision quantitative et dynamique des échanges génétiques, mais exigent une interprétation prudente et contextualisée.
L’avenir de la discipline repose sur la combinaison de la génomique, de l’écologie et de la modélisation, pour appréhender la diversification du vivant comme un processus intégratif et multidimensionnel.
% ==========================

\section{Synthèse et perspectives}
% Discussion générale : ce qu’on retient, les grandes tendances, les questions ouvertes.
L’ensemble des travaux examinés montre que la notion de spéciation a profondément évolué depuis son émergence au début du XXᵉ siècle, passant d’une conception strictement cladistique à une vision plus dynamique et continue, dans laquelle le flux de gènes joue un rôle central.
Là où l’isolement reproductif était autrefois considéré comme absolu, les approches génomiques ont révélé que le transfert de gènes entre lignées divergentes est fréquent. Ce flux peut freiner la différenciation, mais il peut aussi contribuer à la diversification du vivant.
Les exemples issus des canidés nord-américains, de l’espèce humaine et des plantes illustrent la généralité de ces processus à travers les règnes.
Ces avancées ont été rendues possibles par le développement de la bioinformatique et de la génomique, qui ont fourni des tests robustes pour détecter et quantifier l’admixture, tels que les D-statistiques et le test ABBA-BABA.
Cependant, ces approches restent soumises à plusieurs biais, notamment liés à la qualité et à la représentativité des échantillons, parfois difficiles à obtenir. Cela peut conduire à une sous-estimation de la diversité génétique et à des artefacts de couverture affectant les résultats des tests. L’admixture elle-même peut complexifier l’interprétation des scénarios évolutifs.
Malgré ces limites, les perspectives futures demeurent prometteuses : le perfectionnement des modèles, notamment par l'intégration des facteurs environnementaux, et le recours à l'intelligence artificielle ouvrent de nouvelles voies pour la modélisation et la compréhension des processus évolutifs.
En somme, l’étude de l’admixture renouvelle notre compréhension de la spéciation, en montrant qu’elle est moins un événement discret qu’un continuum façonné par les flux génétiques à différentes échelles temporelles et taxonomiques.

% ==========================
\printbibliography

\section*{Licence / License}
\begin{center}
    \includegraphics[width=2.5cm]{figures/cc-by-nc.png}\\[0.5em]
    \textbf{Ce document est distribué sous licence\\
        Creative Commons Attribution–NonCommercial 4.0 International (CC BY–NC 4.0).}\\[0.5em]
    Vous êtes libre de le partager et de l’adapter à condition d’en attribuer la paternité\\
    et de ne pas en faire un usage commercial.\\[0.3em]
    \url{https://creativecommons.org/licenses/by-nc/4.0/}\\[1em]
    \textbf{This document is distributed under the\\
        Creative Commons Attribution–NonCommercial 4.0 International License (CC BY–NC 4.0).}\\[0.5em]
    You are free to share and adapt the material, provided that appropriate credit is given\\
    and it is not used for commercial purposes.\\[0.3em]
    \url{https://creativecommons.org/licenses/by-nc/4.0/}
\end{center}


\end{document}
